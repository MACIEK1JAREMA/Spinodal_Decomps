\documentclass{article}
\usepackage{amsmath,amsthm,amssymb}
\usepackage{bm}
\usepackage[margin=1in]{geometry}
\newcommand{\uvecr}{{\bm{\hat{\textnormal{\bfseries r}}}}}
\DeclareRobustCommand{\uvec}[1]{{%
		\ifcat\relax\noexpand#1%
		% it should be a Greek letter
		\bm{\hat{#1}}%
		\else
		\ifcsname uvec#1\endcsname
		\csname uvec#1\endcsname
		\else
		\bm{\hat{\mathbf{#1}}}%
		\fi
		\fi
}}
\begin{document}

\textbf{Name:} Joe

\textbf{Date:} 26/10/22

\bigskip

We need to calculate the structure factor of the system. This can calculated by finding the Fourier Transform of the order parameter $\phi$ and then finding $\langle \phi(\boldsymbol{k})\phi(\boldsymbol{-k})\rangle$, or $\langle \phi(\boldsymbol{k})\phi(\boldsymbol{k}+\boldsymbol{k_{0}})\rangle$ for some Fourier space element $k_{0}$ (basically the Fourier analogue of calculating the Correlation Function $\langle\phi(\boldsymbol{x})\phi(\boldsymbol{x}+\boldsymbol{r})\rangle$, where $\boldsymbol{k_{0}}$ and $\boldsymbol{r}$ are arbitrary vectors in Fourier and real space). I'm not clear on which equation to use.

\medskip

Once we have the structure factor S, we can plot it against $k$ for various times $t$, and we'll find that they produce different decay curves. For larger $t$, we'll find that the decay curve is steeper. If we scale the $k$ axis by $t^{1/2}$ or $t^{-1/2}$ (I'm not sure which yet), all the curves will fall onto each other.



From this, we should be able to extract the dynamic scaling exponent, which has a value of $z=2$.

\medskip

$\nabla\phi$ can discretised along a square grid with separation $\Delta x$ in each direction.

\begin{equation}
	\nabla \phi \approx \frac{\Delta \phi}{\Delta x^{2}} \frac{\phi_{i+1,j} + \phi{i-1,j} + \phi{i,j+1} + \phi{i,j-1}-4\phi}{\Delta x^{2}}
\end{equation}

\medskip

Also check out "Granular Media.pdf", which is the "PRL 3" article on the Moodle page. Also look at "Growth Laws" in Bray's paper (not sure if section 2.5 or section 7, try both).

\medskip

\rule{\textwidth}{0.4pt}
\end{document}
